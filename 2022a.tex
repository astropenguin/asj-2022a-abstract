% このファイルは日本語用です。
% 次の行は変更しないでください。
\documentclass[ja]{2022a}
%%%%%%%%%%%%%%%%%%%%%%%%%%%%%%%%%%%%%%%%%%%%%%%%%%%%%%%%%%%%%%%%
% 講演者についての情報
\PresenterInfo
%%%%%%%%%%%%%%%%%%%%%%%%%%%%%%%%
% 講演数(半角数字)
{1}
%%%%%%%%%%%%%%%%%%%%%%%%%%%%%%%%
% 氏名
{谷口暁星}
%%%%%%%%%%%%%%%%%%%%%%%%%%%%%%%%
% 氏(ひらがな, 氏名が英字の場合はalphabet)
{たにぐち}
%%%%%%%%%%%%%%%%%%%%%%%%%%%%%%%%
% 名(ひらがな, 氏名が英字の場合はalphabet)
{あきお}
%%%%%%%%%%%%%%%%%%%%%%%%%%%%%%%%
% 所属機関(機関名(◯◯大学、◯◯研究所、など)のみ)
{名古屋大学}
%%%%%%%%%%%%%%%%%%%%%%%%%%%%%%%%
% 会員種別(半角英小文字)
%   a=正会員(一般)
%   b=正会員(学生)
%   c=準会員(一般)
%   d=準会員(学生)
%   e=非会員(一般)〔企画セッションのみ〕
%   f=非会員(学生)〔企画セッションのみ〕
{a}
%%%%%%%%%%%%%%%%%%%%%%%%%%%%%%%%
% 会員番号(半角数字4桁)
%   入会申請中の場合、受付番号(半角英数字8文字)
{5892}
%%%%%%%%%%%%%%%%%%%%%%%%%%%%%%%%
% メールアドレス(半角)
{taniguchi@a.phys.nagoya-u.ac.jp}
%%%%%%%%%%%%%%%%%%%%%%%%%%%%%%%%%%%%%%%%%%%%%%%%%%%%%%%%%%%%%%%%
% 講演についての情報
\PaperInfo
%%%%%%%%%%%%%%%%%%%%%%%%%%%%%%%%
% 記者発表(半角英小文字)
%   申請する場合のみ「y」を記入
{}
%%%%%%%%%%%%%%%%%%%%%%%%%%%%%%%%
% 講演分野(半角)
%  [通常セッション]
%   M=太陽
%   N=恒星・恒星進化
%   P1=星・惑星形成(星形成)
%   P2=星・惑星形成(原始惑星系円盤)
%   P3=星・惑星形成(惑星系)
%   Q=星間現象
%   R=銀河
%   S=活動銀河核
%   T=銀河団
%   U=宇宙論
%   V1=観測機器(電波)
%   V2=観測機器(光赤外・重力波・その他)
%   V3=観測機器(X線・γ線)
%   W=コンパクト天体
%   X=銀河形成・進化
%   Y=天文教育・広報普及・その他
{V1}
%%%%%%%%%%%%%%%%%%%%%%%%%%%%%%%%
% 講演形式(半角英小文字)
%   a=口頭講演
%   b=ポスター講演(口頭有)
%   c=ポスター講演(口頭無)
{a}
%%%%%%%%%%%%%%%%%%%%%%%%%%%%%%%%
% キーワード(5つまで)
%   分野Y以外は PASJ keyword list から選択
{instrumentation: detectors}
{methods: data analysis}
{techniques: spectroscopic}
{}
{}
%%%%%%%%%%%%%%%%%%%%%%%%%%%%%%%%
% 題名
{DESHIMA 2.0: Development overview of the 220--440~GHz integrated superconducting spectrometer and the planned scientific observation campaign on ASTE}
%%%%%%%%%%%%%%%%%%%%%%%%%%%%%%%%
% 氏名及び所属(複数の場合は「, 」で区切)
{
    % Nagoya University
    A. Taniguchi,
    T. J. L. C. Bakx,
    K. Matsuda,
    Y. Tamura (Nagoya),
    % Kitami Institute of Technology
    T. Takekoshi (Kitami),
    % Leiden University
    P. P. van der Werf (Leiden),
    % NAOJ
    R. Kawabe,
    T. Oshima (NAOJ),
    % Toho University
    T. Kitayama (Toho),
    % TU Delft
    J. J. A. Baselmans,
    S. Brackenhoff,
    B. T. Buijtendorp,
    S. Dabironezare,
    A. Endo,
    M. Gouwerok,
    S. H\"{a}hnle,
    K. Karatsu,
    N. Llombart,
    A. Pascual Laguna,
    M. Rybak,
    D. J. Thoen (TU Delft),
    % SRON
    H. Akamatsu,
    R. Huiting,
    V. Murugesan,
    S. J. C. Yates (SRON),
    % University of Tokyo
    T. Ishida,
    K. Kohno (UTokyo),
    % DESHIMA team
    and DESHIMA team
}
\begin{document}
%%%%%%%%%%%%%%%%%%%%%%%%%%%%%%%%%%%%%%%%%%%%%%%%%%%%%%%%%%%%%%%%
% 本文開始
%%%%%%%%%%%%%%%%%%%%%%%%%%%%%%%%%%%%%%%%%%%%%%%%%%%%%%%%%%%%%%%%
Integrated superconducting spectrometer (ISS) technology will enable ultra-wideband submillimeter spectroscopy for uncovering the dust-obscured cosmic star formation and galaxy evolution over cosmic time.
Here we present the current status of DESHIMA 2.0 (Taniguchi et al. arXiv:2110.14656), an ISS that will observe the 220--440~GHz band ($z=3.3\textrm{--}7.6$ for [C~II] 158~$\mu$m) with a resolution of $F / \Delta F \simeq 500$ in a single shot.

As a successor to DESHIMA 1.0 (332--377~GHz band), we upgraded the wideband chip design and the quasi-optical system.
We fabricated the first chip in mid 2021 and the filter response measurement shows that it merits telescope observations.
For better observation efficiency, we also developed a fast sky-position chopper and a data-scientific sky-noise removal method.
With all the upgrades, we expect to detect ($\textrm{S/N}>5$) the [C~II] line of a bright dusty star-forming galaxy (DSFG; $L_{\mathrm{IR}} = 3 \times 10^{13} L_{\odot}$) in an eight-hour ASTE observation.

A three-month scientific observation campaign is planned on ASTE in 2022, including a spectroscopic survey toward high-$z$ DSFGs and a mapping of a galaxy cluster to detect the thermal Sunyaev-Zel'dovich effect signal.
%%%%%%%%%%%%%%%%%%%%%%%%%%%%%%%%%%%%%%%%%%%%%%%%%%%%%%%%%%%%%%%%
% 本文終了
%%%%%%%%%%%%%%%%%%%%%%%%%%%%%%%%%%%%%%%%%%%%%%%%%%%%%%%%%%%%%%%%
\end{document}
%%%%%%%%%%%%%%%%%%%%%%%%%%%%%%%%%%%%%%%%%%%%%%%%%%%%%%%%%%%%%%%%
